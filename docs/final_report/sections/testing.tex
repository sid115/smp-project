\section{Testing Framework}

The testing framework developed for this project plays a pivotal role in validating the functionality and accuracy of the beat detection algorithm. A comprehensive testing script is integral to this process, enabling the automated comparison of BPM estimates produced by our algorithm against those generated by the well-regarded external tool, \texttt{bpm-tools}.

\subsection{Preparation of Test Audio Files}

Before executing the testing script, it's essential to prepare the audio files to ensure consistency and compatibility. This preparation involves two key steps: trimming WAV files to a uniform length and converting FLAC files to WAV format.

\subsubsection{Trimming WAV Files}

The cutWAVs.sh script trims WAV files to the first 30 seconds, standardizing the duration of audio samples for testing. This ensures that the BPM analysis is performed under consistent conditions across all test files. To use cutWAVs.sh, navigate to the directory containing your WAV files and execute:

\lstset{style=ShellStyle}
\begin{lstlisting}[caption={Trimming WAV Files}, label=lst]
./cutWAVs.sh path/to/wav_files_directory
\end{lstlisting}

\subsubsection{Converting FLAC to WAV}

The flac2wav.sh script converts FLAC files to WAV format, preparing them for analysis. This conversion is crucial for maintaining a uniform input format for the beat detection algorithm. To convert FLAC files, execute:

\lstset{style=ShellStyle}
\begin{lstlisting}[caption={Converting FLAC to WAV}, label=lst]
./flac2wav.sh path/to/flac_files_directory
\end{lstlisting}

\subsection{Testing Script Usage}

After preparing the audio files, the testing script, designed to be executed in a Unix shell, automates the evaluation process across these files. Navigate to the \texttt{src} directory and execute:

\lstset{style=ShellStyle}
\begin{lstlisting}[caption={Executing the Testing Framework}, label=lst]
./testBulk.sh path/to/audio_files_directory
\end{lstlisting}

This script processes each WAV file in the specified directory, comparing our BPM estimate with that obtained from \texttt{bpm-tools}. The results are compiled into \texttt{output\_table.txt}, including the song name, our BPM estimate, and the \texttt{bpm-tools} estimate.

Additionally, testBulk.sh calls plot\_bland\_altman.m, generating a Bland-Altman plot that visually compares our BPM estimates against those from \texttt{bpm-tools}. This plot, saved as bland\_altman\_plot.png, provides a graphical representation of the agreement between the two sets of BPM values, highlighting any systematic bias or variability.

\subsection{Test Cases and Methodology}

The framework was applied to Tidal's Top 100 Germany tracks, offering a diverse range of genres and tempos for comprehensive evaluation. This diversity ensures that the testing encompasses a wide range of musical characteristics, challenging the algorithm's versatility and accuracy.

\subsection{Integration into Development}

The testing script is seamlessly integrated into the development process, enabling continuous validation as enhancements and modifications are made to the algorithm. This integration fosters a development cycle that is iterative and data-driven, ensuring that each modification is supported by empirical evidence of performance improvement.

\subsection{Statistical Analysis}

Following the execution of the testing suite on Tidal's Top 100 Germany, the results underwent statistical analysis to evaluate the algorithm's performance. The analysis revealed an average BPM of 133.49 from our algorithm, compared to 121.313 from \texttt{bpm-tools}, with median values of 140 and 122.846, respectively. The median percent difference stood at 3.96497\%, indicating closely aligned performance with minor discrepancies likely due to inherent differences in algorithmic approaches to BPM detection.

This statistical comparison underscores the reliability and accuracy of our algorithm, closely mirroring that of established tools in the majority of test cases. Such analysis validates the algorithm's efficacy and highlights areas for further refinement and optimization.

\subsection{Conclusion}

Through its comprehensive and automated approach, the testing framework has proven indispensable in assessing the beat detection algorithm's accuracy and reliability. By facilitating direct comparison with established benchmarks, it enables a nuanced understanding of the algorithm's performance, guiding ongoing development efforts and ensuring robustness across a diverse musical spectrum.

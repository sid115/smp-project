\section{Conclusion}

The project's core achievements include the successful design and implementation of a signal processing pipeline. This pipeline preprocesses audio files, extracts relevant features, and applies auto-correlation and peak detection techniques to estimate BPM. The algorithm's performance was rigorously tested against a comprehensive dataset, demonstrating a high degree of accuracy when compared to established tools in the field. Furthermore, a robust testing framework has been developed to enable continuous validation and refinement of the algorithm, ensuring its reliability and effectiveness.

\subsection{Challenges Faced}

During the project, we faced and overcame several challenges. The task involved handling diverse musical structures and tempos, which required adjusting parameters and customizing filtering techniques. Additionally, we optimized the algorithm for computational efficiency while maintaining high accuracy, which posed a significant challenge. We had to carefully balance complexity and performance.

\subsection{Future Work}

Looking ahead, there are several potential avenues for future work to enhance the beat detection system. These may include:

\begin{itemize}
    \item \textbf{Machine Learning Integration:} Incorporating machine learning techniques to improve the algorithm's adaptability and accuracy across a broader range of musical styles and recording qualities.
    \item \textbf{Real-time Processing:} Modifying the algorithm to support real-time beat detection, opening up applications in live music performance and interactive installations.
    \item \textbf{User Interface Development:} Creating a user-friendly interface that allows users to interact with the algorithm more effectively, providing real-time feedback and customizable parameters.
    \item \textbf{Algorithmic Optimization:} Continuing to refine the algorithm to enhance its efficiency and accuracy, including exploring alternative signal processing techniques and optimization algorithms.
\end{itemize}

\subsection{Closing Remarks}

In conclusion, this project has provided us with valuable insights into the field of beat detection. We faced several challenges that significantly contributed to our understanding of the subject, including technical difficulties such as signal processing and algorithmic accuracy assessments. This endeavor has broadened our academic and practical knowledge in audio signal analysis. Although we were unable to conduct a comprehensive comparison against an extensive BPM database due to time constraints, we continue to work on establishing a framework for such analysis in the future. We are grateful for the guidance we have received, which has been crucial in our pursuit of a nuanced understanding of beat detection algorithms.

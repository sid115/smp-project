\section{Configuration and Initial Setup}

To begin using the 'Rhythmic Inside' beat detection system, follow these steps to set up the environment and configure the application to meet your specific needs.

\subsection{Cloning the Repository}

To begin, clone the project repository from GitHub and navigate to the \texttt{src/rhythmic\_inside} directory where the source code is located. In a Unix-compatible shell, use the following commands:

\lstset{style=ShellStyle}
\begin{lstlisting}[caption={Cloning from GitHub}, label=lst]
git clone https://github.com/sid115/smp-project.git
cd smp-project/src/rhythmic_inside
\end{lstlisting}

\subsection{Editing the Configuration File}

The behavior of the project can be customized by modifying the constants defined in the \texttt{config.m} file. Although the default values are appropriate for most situations, you may need to adjust these settings to match specific audio characteristics or analysis preferences. To modify any of the following variables, open \texttt{config.m} in a text editor:

\begin{itemize}
    \item \textbf{CUTOFF\_FREQUENCY} (\textit{integer}, Hz): The cutoff frequency for the low-pass filter is defined in order to remove high-frequency noise that is not relevant to the beat detection process. Acceptable values depend on the audio content, but it is typically set to 300 Hz.
    
    \item \textbf{EXPECTED\_BPM} (\textit{integer}, BPM): Influences the calculation of minimum samples between beats.  Set the expected Beats Per Minute (BPM) to reflect the average tempo of the audio files being analyzed.
    
    \item \textbf{FILTER\_ORDER\_LIMIT} (\textit{integer}): Specifies the maximum order for the Butterworth filter.  A higher order will result in a sharper cutoff, but it may introduce phase distortion. Typically, values range from 1 to 5.
    
    \item \textbf{MIN\_PEAK\_MULTIPLIER} (\textit{integer}): Multiplier for determining the minimum peak height in the signal's energy profile, affecting peak detection sensitivity.
    
    \item \textbf{PLOTS\_PREFIX} (\textit{string}): The file path prefix for saving output plots, which are stored in the \texttt{../assets/plots/} directory by default.
    
    \item \textbf{PASSBAND\_RIPPLE} (\textit{integer}, dB): The allowable ripple in the passband of the Butterworth filter, measured in decibels. A typical value is 3 dB.
    
    \item \textbf{SMOOTHING\_WINDOW\_DURATION} (\textit{float}, seconds): Duration of the smoothing window applied to the signal, in seconds. This parameter affects the granularity of the energy calculation.
    
    \item \textbf{STOPBAND\_ATTENUATION} (\textit{integer}, dB): The required attenuation in the stopband of the Butterworth filter, measured in decibels. Commonly set to 40 dB to ensure significant reduction of unwanted frequencies.
\end{itemize}

After adjusting these variables as needed, save your changes to \texttt{config.m}. The system is now configured and ready for beat detection analysis.

\subsection{Running the Project}

After configuring the project to your preferences, execute the beat detection analysis by running the \texttt{main.m} script in a Unix-compatible shell environment.  If you have not already, navigate to the \texttt{src/rhythmic\_inside} directory of the project and use the following command to start the analysis:

\lstset{style=ShellStyle}
\begin{lstlisting}[caption={Running the project}, label=lst]
octave main.m [path/to/wav]
\end{lstlisting}

Replace the placeholder \texttt{[path/to/wav]} with the actual file path of the WAV file that you want to analyze.  If you do not provide a file path, Octave will prompt you with a file manager dialog, allowing you to select a WAV file interactively. This feature ensures that the project is accessible and easy to use, whether you prefer specifying files via the command line or selecting them through a graphical interface.

The script processes the selected audio file according to the configurations set in \texttt{config.m}, performs beat detection, and outputs the estimated BPM to the console.  This streamlined process enables efficient analysis of audio files, making 'Rhythmic Inside' a practical tool for researchers, musicians, and anyone interested in beat detection technologies.

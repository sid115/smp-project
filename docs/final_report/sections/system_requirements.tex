\section{System Requirements}

To guarantee the successful implementation of the 'Rhythmic Inside' beat detection project, the following system requirements must be fulfilled:

\begin{itemize}
    \item \textbf{Operating System:} To run the scripts and commands associated with this project, a Unix-compatible shell environment is necessary. This requirement includes Linux distributions and macOS. Windows users may need to use a compatibility layer like Windows Subsystem for Linux (WSL) or a virtual machine running a Unix-based OS.
    
    \item \textbf{Octave Installation:} To use the project, you must have GNU Octave installed on your system. We recommend using Octave version 8.4.0, but most recent versions should also be compatible. You can install Octave from the official package repository of your Linux distribution or from the Octave website if you are a macOS user.
    
After installing Octave, make sure to install the 'signal' package as well. To do this, enter the following command in the Octave command line interface:

\lstset{style=MATLABStyle}
\begin{lstlisting}[caption={Installing signal in MATLAB/Octave}, label=lst]
pkg install -forge signal
\end{lstlisting}
    
    \item \textbf{Audio File Format:} The supported primary input format is WAV files in PCM format, which provide the uncompressed audio data required for accurate beat detection analysis.
    
    \item \textbf{FLAC to WAV Conversion:} For users who have audio files in FLAC format, we will provide a script to convert these files to the required WAV format. The script will also trim the audio files to the first 30 seconds to standardize the analysis duration and reduce processing time. Further instructions on how to use the scripts will be provided in the Testing framework section.
\end{itemize}

Before installing and executing the project, it is recommended to verify that all system requirements have been met to ensure optimal performance and reliability.

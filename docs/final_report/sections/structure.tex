\section{Project Structure and Implementation}

The 'Rhythmic Inside' project utilizes a complex beat detection system in MATLAB/Octave to analyze WAV audio files and estimate their Beats Per Minute (BPM). The system is structured across multiple '.m' files, each encapsulating a specific part of the processing pipeline, which collectively contribute to the project's functionality. This section explains the modular architecture and the role of each component within the system.

\subsection{Folder Structure}
The project is divided into three primary directories, which reflect the separation of source code, documentation, and additional assets:
\begin{itemize}
    \item \textbf{src:} This directory contains all the source code and shell scripts required to run the beat detection algorithms. It includes individual '.m' files for each function in the processing pipeline, as well as the main script \texttt{main.m} which manages the execution flow.
    \item \textbf{assets:} This folder contains graphical resources, sample WAV files for testing, and output PDFs generated from the \texttt{docs} directory. It is designed to aid in both testing and demonstrating the project's capabilities.
    \item \textbf{docs:} The LaTeX source code for the project's documentation is stored here. It offers a complete overview of the system, including its design, implementation details, and instructions for usage.
\end{itemize}

\subsection{Implementation Details}
The source code in the \texttt{src/rhythmic\_inside} directory has a modular design, with essential components of the beat detection process implemented as separate functions:

\begin{itemize}
    \item \textbf{preprocess.m:} The audio signal undergoes initial preprocessing, which includes converting it from stereo to mono and applying a low-pass filter to reduce high-frequency noise.
    \item \textbf{config.m:} A configuration script is used to define global parameters that are utilized across various functions. These parameters include filter specifications and threshold values.
    \item \textbf{calculateEnergy.m:} The preprocessed audio is used to compute the energy signal, which highlights variations in signal intensity and facilitates the detection of rhythmic patterns.
    \item \textbf{autoCorrelation.m:} Applies an auto-correlation function to the energy signal to identify periodicities that indicate beats.
    \item \textbf{detectPeaks.m:} The auto-correlated signal is analyzed to detect significant peaks, which may represent potential beats within the audio.
    \item \textbf{calculateBPM.m:} This function calculates the beats per minute (BPM) by analyzing the intervals between detected peaks in the audio signal.
\end{itemize}

The main script, \texttt{main.m}, integrates these components, orchestrating the entire beat detection process from file input through to BPM estimation. It checks for the necessary dependencies, reads the input WAV file, and sequentially invokes the aforementioned functions to process the signal and output the estimated BPM.

\subsection{Modular Design and Flexibility}
By structuring the project with each function in its own file, 'Rhythmic Inside' achieves high modularity, allowing for easy updates, maintenance, and scalability. This design approach not only facilitates the understanding and debugging of individual components but also supports the extension of the project to incorporate new features or algorithms in the future. \\

In summary, the 'Rhythmic Inside' project's organization and implementation strategy enables a robust and efficient beat detection system. The clear separation of concerns, combined with a thoughtful directory structure, ensures that the project is both manageable and adaptable to evolving requirements in the field of digital signal processing.

\documentclass[a4paper,12pt,oneside]{article}
\usepackage{caption}
\usepackage[T1]{fontenc}
\usepackage{fancyhdr}
\usepackage{lmodern}
\usepackage{easyReview}
\usepackage{pgfgantt}
\usepackage{geometry}
\geometry{
	a4paper,
	left=30mm,
	right=20mm,
	top=15mm,
	bottom=10mm,
	heightrounded,
	textheight=245mm,
	textwidth=160mm,
	includeheadfoot,
	headsep=10mm,
	footskip=10mm,
	headheight=14.599pt
}

\def\tightlist{%
  \setlength{\itemsep}{0pt}\setlength{\parskip}{0pt}}

\author{}
\date{}

\begin{document}

\begin{titlepage}
  \centering
  \vspace*{1cm}
  
  {\Large\textbf{Rhythmic Insight}}\\[0.5cm]
  {\large\textit{Automated Beat Detection and BPM Analysis Framework}}\\[2cm]
  
  {\large by Sven Fuchs and Joshua Reichmann}\\[1cm]
  
  \vfill
  \begin{minipage}{0.8\textwidth}
  This project plan outlines the necessary steps to complete the project
  on time. It includes a brief description of the content, solution
  requirements, necessary sub-steps, and a schedule.
  \end{minipage}
  
  \vfill
  
  {\large \today}
\end{titlepage}

\newpage
\pagestyle{fancy}
\fancyhf{}
\fancyhead[R]{\textsl{\leftmark}}
\fancyfoot[C]{\thepage}
\setcounter{page}{1}

\section{Brief description}

The `Rhythmic Insight' project is an academic undertaking that focuses
on developing a MATLAB-based application to analyze audio tracks and
determine their beats per minute (BPM). The project is primarily
concerned with extracting musical rhythm from integrated music files,
using a range of digital signal processing techniques such as Discrete
Wavelet Transforms (DWT) and auto-correlation.

The aim of this project is to investigate automated beat detection and
demonstrate the suitability of MATLAB as a platform for conducting
musical analyses. Users can input their music files into the
application, which will analyze the audio signals to identify potential
beat progressions and calculate the tempo of the track. The software is
designed to process audio input in segments, applying a series of
mathematical operations to isolate the rhythmic components and determine
the beats per minute (BPM) for each segment.

A crucial aspect of this project is the utilization of median
calculation to derive the overall BPM across all measured segments. This
method is expected to improve the reliability and accuracy of the BPM
measurement, especially in tracks with varying tempos or complex
rhythmic structures. Additionally, the application will offer users a
graphical representation of the auto-correlation of the processed
signal, providing a visual interpretation of the rhythmic patterns and
tempo fluctuations.

The project plan will outline the phases of design, implementation,
testing and deployment of the software. It will also address potential
challenges and propose strategies to mitigate them. The project aims to
deliver an efficient and functional tool for BPM detection, contributing
to the field of musical analysis and demonstrating the capabilities of
MATLAB for signal processing applications.

\section{Aims of the project}

The final aims of the `Rhythmic Insight' project should encapsulate the
broad goals of the software while remaining attainable for a student
group. The following are some general aims for the software:

\begin{enumerate}
\def\labelenumi{\arabic{enumi}.}
\item
    \textbf{Accurate BPM Detection}: The aim of this project is to develop an
  algorithm that can precisely detect the BPM of various music genres.
  The target is to achieve a high level of accuracy, which can be
  considered \emph{appropriate} for academic projects. It is important
  to note that the goal is not perfection, but rather a consistent and
  reliable analysis.
\item
  \textbf{Handling of Varied Musical Structures}: The software must be capable of
  processing both simple and complex rhythmic structures, including
  straightforward beats and more intricate patterns found in genres with
  fluctuating tempos and intricate rhythms.
\item
  \textbf{User-Friendly Interface}: Design a user interface that is intuitive and
  easy for users to navigate. The interface should allow users to upload
  audio files, initiate analysis, and view results in a clear and
  understandable format.
\item
  \textbf{Efficient Processing}: The objective is for the software to efficiently
  process audio files, optimizing for reasonable analysis times without
  sacrificing accuracy. It should be able to analyze a standard-length
  track in a practical time frame for users.
\item
  \textbf{Adaptability and Scalability}: The software should be designed with
  adaptability in mind, allowing for future enhancements and additions.
  It should be structured in a way that new methods or improvements can
  be integrated without extensive overhauls.
\item
  \textbf{Educational Value}: The project should have educational value by
  providing clear insights into digital signal processing, rhythm
  analysis, and software development. Thorough documentation should
  explain the methods, algorithms, and design choices made.
\item
  \textbf{Robustness and Reliability}: The software should be designed to be
  robust, capable of handling various input types and qualities with
  minimal errors or failures. It should consistently produce accurate
  results across multiple runs on the same input.
\item
  \textbf{Testing and Validation}: Conduct comprehensive testing on a variety of
  music tracks to validate the software's performance. Document the
  testing process and results to provide transparency and understanding
  of the software's capabilities and limitations.
\end{enumerate}

These aims provide a broad framework for developing the `Rhythmic
Insight' software. The project aims to create a useful, educational, and
reliable tool for detecting BPM and analyzing music.

\section{Roadmap}

The `Rhythmic Insight' project comprises a series of progressively
complex objectives aimed at developing a comprehensive understanding of
musical tempo and rhythm through signal processing techniques. The
following are the outlined objectives for the project:

\begin{enumerate}
\def\labelenumi{\arabic{enumi}.}
\item
  Basic Beat Detection in Simple Rhythms:
  \begin{itemize}
  \tightlist
  \item
    \textbf{Objective}: To develop an initial capability of the software
    to detect the BPM of tracks with simple, consistent beats such as
    those found in techno music.
  \item
    \textbf{Approach}: Implement an algorithm to analyze the energy of the
    audio signal and identify peaks corresponding to beats. This will
    involve setting a threshold to distinguish between the beats and other
    parts of the audio signal.
  \end{itemize}

\setcounter{enumi}{1}
\item
  Enhancement of Beat Detection Accuracy:
  \begin{itemize}
  \tightlist
  \item
    \textbf{Objective}: To refine the beat detection process for greater
    accuracy, enabling the software to detect beats reliably across a
    wider range of simple rhythmic patterns.
  \item
    \textbf{Approach}: Improve the energy analysis algorithm to adapt to
    different energy levels and incorporate a smoothing mechanism to
    reduce false positives and negatives.
  \end{itemize}

\setcounter{enumi}{2}
\item
  Introduction of Auto-Correlation for Complex Rhythms:
  \begin{itemize}
  \tightlist
  \item
    \textbf{Objective}: To extend the software's capabilities to handle
    more complex rhythms, such as those found in jazz or classical music,
    which may have varying tempos and more intricate beat structures.
  \item
    \textbf{Approach}: Implement an auto-correlation function to analyze
    the time intervals between beats, accommodating for variations and
    irregularities in the rhythm.
  \end{itemize}

\setcounter{enumi}{3}
\item
  Multi-Level Signal Decomposition:
  \begin{itemize}
  \tightlist
  \item
    \textbf{Objective}: To analyze tracks with multiple instruments and
    layered rhythms, extracting and isolating the beat from a complex
    mixture of sounds.
  \item
    \textbf{Approach}: Use Discrete Wavelet Transforms (DWT) to decompose
    the audio signal into various frequency components and analyze each
    one for rhythmic information.
  \end{itemize}

\setcounter{enumi}{4}
\item
  Temporal Dynamic Analysis:
  \begin{itemize}
  \tightlist
  \item
    \textbf{Objective}: To detect changes in tempo and rhythm over time
    within a single track, identifying transitions and variations.
  \item
    \textbf{Approach}: Implement a sliding window technique to perform
    beat detection over short, overlapping segments of the track, allowing
    the detection of changes in tempo and rhythm throughout the piece.
  \end{itemize}

\setcounter{enumi}{5}
\item
  User Interface Development:
  \begin{itemize}
  \tightlist
  \item
    \textbf{Objective}: To create an intuitive and user-friendly interface
    that allows users to easily input their audio files and understand the
    analysis provided by the software.
  \item
    \textbf{Approach}: Develop a (\emph{optional}: graphical) user
    interface (GUI) that provides visual feedback, such as waveforms and
    tempo graphs, and allows for easy navigation and operation of the
    software.
  \end{itemize}

\setcounter{enumi}{6}
\item
  Validation and Testing:
  \begin{itemize}
  \tightlist
  \item
    \textbf{Objective}: To ensure the reliability and accuracy of the BPM
    detection across a variety of music genres and recording qualities.
  \item
    \textbf{Approach}: Conduct extensive testing using a diverse set of
    audio tracks, comparing the software's BPM output with manually
    verified values and adjusting the algorithms as necessary to improve
    accuracy.
  \end{itemize}
\end{enumerate}

The `Rhythmic Insight' project aims to create a versatile and reliable
tool for analyzing musical tempo and rhythm, contributing valuable
insights and methodologies to the field of digital music analysis.

\section{Schedule}

To manage and track the progress of the `Rhythmic Insight' project, a
Gantt chart is presented below. The chart outlines the timeline, key
milestones, and dependencies of various tasks, providing a visual
roadmap for the project's development from inception to completion.

\begin{ganttchart}[hgrid, vgrid, x unit=0.6cm]{1}{13}
\gantttitle{Calendar Weeks}{12}
\ganttnewline
\gantttitle{50}1
\gantttitle{51}1
\gantttitle{52}1
\gantttitle{01}1
\gantttitle{02}1
\gantttitle{03}1
\gantttitle{04}1
\gantttitle{05}1
\gantttitle{06}1
\gantttitle{07}1
\gantttitle{08}1
\gantttitle{09}1
\gantttitle{10}1
\ganttnewline
\ganttbar{Project Plan (21.12.) - B}{1}{1}
\ganttnewline
\ganttbar{Brief Description - S}{1}{1}
\ganttnewline
\ganttbar{Aims - S}{1}{1}
\ganttnewline
\ganttbar{Roadmaps - J}{1}{1}
\ganttnewline
\ganttbar{Schedule - J}{1}{1}
\ganttnewline
\ganttbar{Development - B}{3}{3}
\ganttnewline
\ganttbar{Basic Beat Detection - S}{3}{3}
\ganttnewline
\ganttbar{Accuracy Enhancement - J}{3}{3}
\ganttnewline
\ganttbar{Auto-Correlation - J}{5}{5}
\ganttnewline
\ganttbar{Multi-Level Signal Decomp. - S}{5}{5}
\ganttnewline
\ganttbar{Temporal Dynamic Analysis - J}{7}{7}
\ganttnewline
\ganttbar{UI - S}{7}{7}
\ganttnewline
\ganttbar{Validation - B}{9}{9}
\ganttnewline
\ganttbar{Documentation - B}{4}{4}
\ganttnewline
\ganttbar{Interim Report (19.01.) - B}{6}{6}
\ganttnewline
\ganttbar{Final Report (07.03.) - B}{11}{11}
\ganttnewline
\ganttbar{Backup - B}{13}{13}
\ganttnewline
\end{ganttchart}

\hfill \textit{B = both, J = Joshua, S = Sven}
\newline

The Gantt chart above serves as a strategic planning tool to ensure that
all project components are addressed in a timely and orderly manner. It
will be instrumental in monitoring progress, managing resources, and
achieving the set objectives within the designated time frame. Regular
reviews and updates to the Gantt chart will be conducted to reflect any
changes or developments as the project advances.

\end{document}
